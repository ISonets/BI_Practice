 \documentclass{article}
 \usepackage[utf8]{inputenc}
 \usepackage{graphicx}
 \graphicspath{ {./images/} }
 \usepackage{amsmath}
 \usepackage{hyperref}
 %\usepackage[english]{babel}
 \usepackage{pdfpages}
 \usepackage{subfig}
 \usepackage{chemfig}
 \usepackage[left=2cm,right=1cm, top=2cm,bottom=2cm,bindingoffset=0cm]{geometry}
 \renewcommand{\normalsize}{\fontsize{14}{18pt}\selectfont}
 \newcommand{\specialcell}[2][c]{%
 	\begin{tabular}[#1]{@{}c@{}}#2\end{tabular}}
 \renewcommand{\normalsize}{\fontsize{14}{18pt}\selectfont}
 
 \title{  }
 \author{ Ignat Sonets, Kamilla Faizullina}
 \date{\empty}
 
 \begin{document}
 	
 	\catcode`\_=\active
 	\catcode`\^=\active
 	
 	\maketitle
 	
 	\textbf{Abstract.}  In this work, we follow a RepSeq data analysis tutorial 
 	
 	\section{Introduction}
 
 
 \section{Results}
 
   	\begin{table}[h!]
 	\centering
 	\begin{tabular}{|c|c|c|c|c|}
 		\hline
 	sample &	donor &	status &	subset&	phenotype \\ \hline
 	s1	& D1	&	CMV-		& CD8 &	memory \\ \hline
 	s2	&D2		&CMV+	&	CD8	& naive \\ \hline
 	s3	&D1		&CMV+	&	CD8	& memory\\ \hline
 	s4	&D2		&CMV-	&	CD4	& memory\\ \hline
 	s5	&D2		&CMV+	&	CD8	& memory\\ \hline
 	s6	&D2		&CMV-	&	CD4	& memory\\ \hline
 	s7	&D1		&CMV+	&	CD8	& memory\\ \hline
 	s8	&D1		&CMV-	&	CD4	& memory\\ \hline
 	s9	&D2		&CMV+	&	CD8	& memory\\ \hline
 	s10	&D2		&CMV-	&	CD4	& memory\\ \hline		
 	s11	&D2		&CMV+	&	CD8	& memory\\ \hline
 	s12	&D2		&CMV+	&	CD8 &	memory\\ \hline
 	s13	&D2		&CMV+	&	CD8	&naive\\ \hline
 	s14	&D2		&CMV+	&	CD8	&memory\\ \hline
 	s15	&D2		&CMV-	&	CD8	&naive\\ \hline
 	s16&	D2	&	CMV-	&	CD4	&naive\\ \hline
 		
 		\hline	 
 		
 	\end{tabular}
  	\caption{ Summarizing results  }
  \label{tab:var}
\end{table}



 	\begin{thebibliography}{9}
 		
 		
 		
 
 	\end{thebibliography}
 	
 	
 	
 	
 \end{document}
 
 
 